\cleardoublepage
\phantomsection
\addcontentsline{toc}{chapter}{Введение}
\chapter*{Введение}
\label{chap:introduction}
Идея построения машины, обладающей интеллектом, способной искать и принимать оригинальные решения, базируется на чём-то большем, чем математическая модель, писал А. Тьюринг, — на чувствах удовлетворения и горя, факторе внезапности, чтобы обучаться подобно ребёнку. ~\cite{intelligent_machinery} Связь мышления с эмоциями подтверждает и объясняет М. Мински в труде об искусственном эмоциональном интеллекте "The emotion machine".~\cite{minsky2007}


Эмоция, по соглашению психологов и нейробиологов, это адаптивное согласованное изменение в нескольких физиологических системах (соматических и нервных) в ответ на внешний или внутренний стимул.\cite{plutchik2001} Большую часть прошлого века разум и эмоции было принято противопоставлять. На настоящее же время нейробиологами многократно доказано, что рациональное, логическое мышление и эмоции тесно переплетены и неразделимы.\cite{Brosch, ziemke-role, damasio2008, damasio1994}


Эволюционно эмоции обеспечивают выживание, предупреждая  организм об угрозах, регулируют поведение, направляя организм на удовлетворение актуальных потребностей.\cite{whoneedsemotions} Эмоции контролируют внутреннее психологическое состояние, управляют принятием решений.\cite{roleofemotions} Эмоции участвуют в процессах обучения, памяти, управляют вниманием и восприятием, механизмы сознания и эмоций связаны на всех уровнях.\cite{Phelps2006} Социальные взаимодействия реальных людей с субъектом без эмоций затруднены и непродуктивны, поскольку эмоции регулируют адекватность в общении, вычисляют наиболее подходящий к ситуации и настроению собеседника ответ. Эмпатия, жалость и любовь это сложные социальные эмоции, без наличия которых взаимодействие с людьми просто потенциально опасно. Эмоции нужны для гибкости, ускоренной адаптации и развития. Исходя из этих фактов, во-первых, понимание человеческих эмоций это необходимый шаг для понимания человеческого сознания. Во-вторых, искусственный интеллект и когнитивные технологии неосуществимы без реализации эмоций машины, вычислительных эмоций.\cite{affectivecomputing}


Исследования по воспроизведению эмоций в вычислительных машинах называются Affective Computing.\cite{affectivecomputingchallenges} При наличии различных подходов к этой проблеме, на данный момент не существует таких, которые бы опустились по уровню абстракции до уровня нейробиологических процессов в головном мозге — при этом сама идея логична, ведь именно такой уровень мыслительных процессов теоретически сопоставим с вычислительными процессами. Моделированием на этом уровне занимаются в лаборатории машинного понимания Высшей школы ИТИС, где разрабатывается когнитивная архитектура NeuCogAr (Neural Cognitive Architecture). Данная работа — завершение двухлетнего проекта по реализации трёх систем нейромодуляторов для воспроизведения 8 базовых эмоций.


Объектом исследования в дипломной работе являются вычислительные эмоции в биологически реалистичной когнитивной архитектуре. Предмет исследования – принципы влияния нейромодуляторов норадреналин, серотонин, дофамин, ацетилхолин, ГАМК и глутамат на мозг и друг на друга; перенос этих механизмов в область вычислительных технологий.

Цель дипломной работы — интегрировать три системы нейромодуляторов в когнитивной архитектуре NeuCogAr для реализации 8 эмоциональных состояний на биологически реалистичном количестве нейронов. Для достижения поставленной цели необходимо решить следующие задачи:
\begin{itemize}
\item Реализовать систему распространения нейромодулятора норадреналин на NEST Simulator (инструменте моделирования импульсных нейронных сетей);
\item Интегрировать программные реализации систем норадреналина, серотонина и дофамина;
\item Исследовать воздействие трёх моноаминовых нейромодуляторов друг на друга в мозге млекопитающего;
\item Воспроизвести это воздействие на вычислительной нейронной сети в 500 000 нейронов (биологически реалистичное количество для мыши);
\item Запустить полученную систему в разных конфигурациях, согласно модели Хьюго Левхейма, фиксируя 8 различных эмоций машины.
\end{itemize}

Основным инструментом в дипломной работе является симуляционное ядро NEST Initiative, находящееся под контролем пакета языка Python. Его основным пользовательским интерфейсом является PyNEST, который удобен в использовании и хорошо взаимодействует с библиотеками языка Python, например, matplotlib, numpy, pylab.