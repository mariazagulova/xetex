\chapter{Современное состояние исследований в области эмоциональных вычислений.}
\label{chap:literature_review}
Современные социальные роботы могут имитировать эмоции в соответствии с выражением лица собеседника: андроид София из Hanson Robotics показывает 60 эмоций; андроид ERICA из Hiroshi Ishiguro Laboratories имитирует выражение лица собеседника, если в её сценарии нет ответной эмоции на его реплику \cite{erica}; социальный робот Pepper распознаёт эмоцию собеседника по выражению лица, тону голоса и ключевым словам, реагирует согласно заранее заданным сценариям.~\cite{meetpepper} Все они известны именно как «эмоциональные», обладающие «эмоциональным движком», создают иллюзию наличия эмпатии, но не являются когнитивными технологиями и не испытывают эмоций.


Иной подход к эмоциям показан С. Бризейл: эмоции как система мотивации, эмоции как реакция на то, достигнуты ли цели и стремления организма.\cite{Breazeal2003} Её робот KISMET обладает стремлениями общаться с людьми и наблюдать за яркими игрушками, периодически засыпая — как человеческий младенец. Если жизненные цели робота не выполняются, то в нём нарастают негативные эмоции (грусть, раздражение) как средство мотивации к их выполнению. По достижению цели робот испытывает положительные эмоции, которые служат сигналом успокоиться и отдохнуть. Если же стимуляция робота перевыполнена, к нему подошли слишком близко, пугают его — эмоции робота вновь станут негативными (страх), мотивирующими избежать нежелательных стимулов, например, отвернуться, закрыть глаза.


Модель эмоций, мотивирующих робота KISMET, опирается на «колесо эмоций», модель, созданную Р. Плутчиком, группирующую в пары 8 базовых эмоций: грусть — радость, страх — ярость, отвращение — приятие, удивление — разочарование. Процесс влияния эмоций на организм Плутчик описывает как последовательность событий: стимулирующее событие — ожидание реакции — ощущение эмоции — физиологическое возбуждение — импульс к действию — совершение действия — влияние действия.\cite{plutchik2001} С учётом этой модели в 2012 году Э. Камбрия и др. разработали биологически-инспирированную и опирающуюся на психологию модель представления любой эмоции в пространстве 4 измерений: внимание, чувствительность, удовольствие, уверенность.\cite{hourglass}


На междисциплинарном симпозиуме AAAI по эмоциональным архитектурам были представлены взгляды на моделирование эмоций с различных сторон. Физиологическая модель с максимально упрощёнными частями организма: дыхательной системой, сердцем и пр.\cite{aaai1} Многоуровневая модель, учитывающая логические рассуждения, убеждения и систему Брамса о человеческом поведении.\cite{aaai2} Простые эмоции для выстраивания победной тактики роботической рукой в игре «камень — ножницы — бумага», учитывающие удовольствие от победы и злость от поражения.\cite{aaai3} Моделирование эмоций игрока обучающей игры с помощью динамических байесовских сетей.\cite{aaai4} Эмоциональная архитектура для робота Казимиро на основе BDI (архитектуры убеждений, желаний и намерений).\cite{aaai5} Эмоциональная архитектура на основе парадигмы принятия решений по нескольким критериям с использованием нечёткой логики.\cite{aaai6}


Когнитивист Т. Цимке не отделяет эмоции от всех процессов познания в целом, его эмоционально-когнитивные архитектуры всегда базируются на нейрофизиологии, моделировании участков мозга в соответствии с реальными биологическими функциями. Например, робот-крыса с чувствительными усиками для проекта ICEA имеет мозжечок, гиппокамп, кору мозга, базальные ядра, и каждый этот участок имеет свою функцию, связанную с памятью либо мотивацией. В этих архитектурах эмоциональные состояния несут конкретные функции для выживания, поддержания активности, памяти, предсказаний и планирования.\cite{ziemkeflexibility}


Отдельно идёт развитие автономных военных роботов, которых исследователи учат принимать решения, связанные с моралью и этикой, в вопросах доверия, жертвенности и вины.\cite{moralrobots} Речь также идёт об эмоциях, классифицированных Дж. Хальдтом как «моральных»: уважение, благодарность, сострадание, презрение, стыд, вина и ярость.\cite{moralemo} Они реализованы не биологически реалистично, а подсчётом сложных алгоритмов, принимающих решение поступенчато, не исключая и законов Азимова.\cite{militaryrobotics}