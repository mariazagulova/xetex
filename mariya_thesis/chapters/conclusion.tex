\cleardoublepage
\phantomsection
\addcontentsline{toc}{chapter}{Заключение}
\chapter*{Заключение}
\label{chap:conclusion}
Дипломная работа посвящена актуальной задаче реализации биологически реалистичных эмоциональных состояний на уровне вычислительных процессов, сопоставимых с мыслительными процессами в мозге млекопитающего. В рамках проведённых исследований выполнено:
\begin{itemize}
\item Реализована система распространения нейромодулятора норадреналин;
\item Интегрированы программные реализации систем норадреналина, серотонина и дофамина;
\item Проведены эксперименты по исследованию воздействия трёх моноаминовых нейромодуляторов друг на друга, на связи между нейронами и на нейронную активность в разных областях мозга;
\item Это воздействие воспроизведено на вычислительной нейронной сети в 500 000 нейронов;
\item Полученная система запущена в разных конфигурациях, согласно модели Хьюго Левхейма, итогом чего становятся зафиксированные и проанализированные 8 различных эмоций машины.
\end{itemize}
	
	
Результаты проведённых вычислительных экспериментов показывают, что реалистичная симуляция функций биологических нейромодуляторов осуществима на вычислительных машинах, что становится новым шагом в области когнитивных технологий. Достигнута поставленная для дипломной работы цель: получена общедоступная, универсальная и эффективная модель вычислительных эмоций в когнитивной архитектуре NeuCogAr. Поскольку модель работает на уровне вычислительных процессов, её возможно перенести на аппаратное обеспечение, для объединения систем или модулей системы без необходимости перепрограммировать всю архитектуру. Реализованные в работе базовые эмоции имеют исключительно функциональные задачи: улучшения качества реагирования на внешние события, приспособляемости к внешним условиям, обучаемости и принятия решений.


Итогом исследования стало написание трёх статей по тематике работы:

\begin{itemize}
\item «The Implementation of Noradrenaline in the NeuCogAr Cognitive Architecture» в соавторстве с М. Талановым, Б. Пинусом, J. Vallverdu и др. для участия в IX международной конференции по продвинутым когнитивным технологиям и приложениям COGNITIVE-2017.
\item «NeuCogAr: how to make a machine feel emotions. Neuromodulating cognitive architecture for mammalian emotions simulation» в соавторстве с Хьюго Левхеймом, М. Талановым, J. Vallverdu, Б. Пинусом, Ф. Гафаровым, С. Дистефано, А. Хасьяновым, А. Тощевым, Е. Магидом, Р. Гайсиным и др.
\item «Extending biomimetic cognitive architecture NeuCogAr with noradrenaline model» в соавторстве с М. Талановым, Б. Пинусом, J. Vallverdu, С. Дистефано, Ф. Гафаровым и А. Леухиным для журнала «Journal of Healthcare Engineering».
\end{itemize}


Проект собрал международную команду исследователей, привлёк интерес институтов и компаний в областях медицины, экономики, педагогии, инженерии. Кроме ориентации на неизбежный перенос модели на аппаратное обеспечение, в область робототехники, существуют перспективы исследований, для которых достаточно проведения программных симуляций. Биологически реалистичное количество нейронов делает эти симуляции способными отвечать на запросы маркетологов, симулируя реакцию мозга на рекламные компании, и медиков, симулируя поведение мозга при травмах. Модель будет расширена от эмоций до многих других столь же важных процессов в мозге. Но и работа над вычислительными эмоциями не закончена реализацией базовых эмоций, поскольку восьми эмоций не будет достаточно социальному роботу будущего, данная работа --- лишь первый шаг.