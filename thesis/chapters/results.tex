\chapter{Симуляция базовых эмоций в трёхмерной модели}
\label{chap:results}
В системе, полученной из трёх совмещённых систем нейромодуляторов на языке Python, с общим словарём областей мозга, выделенными для них нейронами в реалистичных пропорциях и настроенными связями, были проведены 8 симуляций на 500 000 нейронов с различными конфигурациями:

1. Эмоция «тоска/горе». Требует высокого уровня норадреналина при низких уровнях дофамина и серотонина. Генераторы спайков подключены к ядрам солитарного тракта, ядрам paragigantocellular, латерально-спинному ядру покрышки и perirphinal cortex с 200 по 600 мс.

/////////////////////////////////////здесь будут картинки и объяснения картинок

2. Эмоция «презрение/отвращение». Требует высокого уровня серотонина при низких уровнях дофамина и норадреналина. Генераторы спайков с 300 по 600 мс подключены к серотониновым очагам в префронтальной коре, моторной коре и ядрах шва (дорсальном и среднем).

/////////////////////////////////////здесь будут картинки и объяснения картинок

3. Эмоция «испуг/ужас». Формируется при высоком уровне дофамина и низких уровнях серотонина и норадреналина. Генераторы спайков подключены к моторной коре, чёрной субстанции pars compacta, миндалевидному телу и вентральной области покрышки с 200 по 600 мс.

/////////////////////////////////////здесь будут картинки и объяснения картинок

4. Эмоция «радость/удовольствие». Требует одновременного существования спайковой активности в центрах дофамина и серотонина (чтобы серотонин не подавил дофамин), для чего были созданы центры ---------------- в --------. Концентрация норадреналина должна оставаться низкой

5. Эмоция «злость/ярость». Возникает при высоких уровнях норадреналина и  дофамина, а активность одного из них всегда влечёт активность другого. С 100 по 600 мс генерируются спайки на ядрах солитарного тракта, ядрах paragigantocellular, латерально-спинном ядре покрышки и perirphinal cortex. На 400 мс к ним добавляются генераторы на моторной коре, чёрной субстанции pars compacta, миндалевидном теле и вентральной области покрышки, вместе они работают до 600 мс.

6. Эмоция «удивление». Высокие уровни серотонина и норадреналина, норадреналин борется за сохранение своего возбуждающего влияния, но не в полной мере сохраняет его под ингибирующим действием серотонина. К генераторам в префронтальной коре, моторной коре, дорсальном и среднем ядрах шва (работают с 300 по 600 мс) добавляются на 400 секунде генераторы на ядрах солитарного тракта, ядрам paragigantocellular, латерально-спинному ядру покрышки и perirphinal cortex, вместе они работают до 600 мс.

7. Эмоция «стыд/унижение». Пониженные уровни всех трёх нейромедиаторов.

8. Эмоция «интерес/азарт». Высокие уровни всех трёх нейромедиаторов. Генераторы спайков в центрах дофамина (моторной коре, чёрной субстанции pars compacta, миндалевидном теле и вентральной области покрышки) работают начиная с 200 мс, генераторы в центрах серотонина (префронтальная кора, моторная кора, дорсальное и среднее ядра шва) подключаются на 300 мс, генераторы активности норадреналина (на ядрах солитарного тракта, ядрах paragigantocellular, латерально-спинном ядре покрышки и perirphinal cortex) подключаются на 400 мс.